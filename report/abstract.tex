\begin{abstract}

A cellular automaton (CA) is an \textit{N}-dimensional lattice, where each location in the lattice can be in one of \textit{k} possible states. 
At each time step, the state of each location is updated according to a rule that maps a surrounding neighborhood to 
a new state for that location. Despite their simplicity, CAs can solve non-trivial problems. In this paper we investigate the capacity of 
1-dimensional CAs to solve the `$\rho = \frac{1}{2}$' problem, i.e., the problem of taking an initial binary string and iteratively updating 
its states so that the final outcome correctly classifies the density of \texttt{1}s in the initial string: If the string has more \texttt{1}s than 
\texttt{0}s, the final lattice should be all \texttt{1}s, and should be all \texttt{0}s otherwise. A genetic algorithm with mutation and crossover 
is used to search for solutions. We show that (i) CA that have access to more information regarding the string they are classifying perform 
better than those with less information despite having a larger space of possible solutions, and (ii) well-performing CA exhibit `mutational 
robustness' in the sense that the CA continues to perform well despite random mutations. Finally, we modify the genetic algorithm to utilize a dynamic fitness 
function that gradually encourages solutions to harder initial strings and discourages reliance on previously found 
solutions; We show that this `biased fitness' approach leads to better-preforming solutions in comparison to the unbiased version.
\end{abstract}
