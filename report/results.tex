\section{Results} \label{results}

For the results reported here, the CA population is 100, selection is tournament with elitism, offspring CA populations are created with both 
mutation and crossover, the IC length is 121, the size of the IC test set is 100, and CAs are run for a maximum of 
300 iterations per IC over 50 generations. See section \ref{sec:methods} 
for more information regarding methods.

\subsection{Comparing neighborhoods} \label{sec:2_1}
Our first goal was to compare rules with different neighborhoods. In particular, there is a tradeoff between the size of the search space and the 
the amount of information available for purposes of setting a location state. Larger neighborhoods have rules that can `see' more of the source IC, and 
hence make a more informed decision about how to set the output bit, but this comes at a cost of increasing the size of the search space by multiple 
orders of magnitude ($2^{32}$ vs. $2^{128}$).

Figures~\ref{fig:waterfall_1} and \ref{fig:waterfall_2} show two typical CA runs. The first is an example of a rule that does not converge, while the second 
is a rule that successfully converges within 300 iterations.

\begin{figure}
\begin{center}
\includegraphics[width=0.8\linewidth]{figures/non_converging_radius2.png}
\caption{Waterfall plot of a radius 2 rule that fails to converge within 300 iterations. The top row is the initial condition, and each successive row is the 
result of applying the rule to the row above it.}
\label{fig:waterfall_1}
\end{center}
\end{figure}

\begin{figure} [H]
\begin{center}
\includegraphics[width=0.8\linewidth]{figures/converging_radius2.png}
\caption{Waterfall plot of a radius 2 rule that successfully classifies an initial condition.}
\label{fig:waterfall_2}
\end{center}
\end{figure}

After running the GA for 50 runs, for both $r = 2$ and $r = 3$, we evaluated the elite rules from the final generation of every run against a test set of 10k ICs, and ranked them by their fitness score. Figure~\ref{fig:r2_r3_rho} plots the performance of the best performing $r = 2$ and $r = 3$ rules. As the figure shows, 
the best $r = 3$ outperforms the best $r = 2$ rule.

\begin{figure}
\begin{center}
\includegraphics[width=\linewidth]{figures/lambda_correctness_plot.eps}
\caption{The best performing elite rules from 50 GA runs each for $r = 2$ and $r = 3$. As expected, both rules perform well for extreme values 
of $\rho$, but performance drops as $\rho$ approaches the difficult cases around $\frac{1}{2}$. Furthermore, the $r = 3$ rule performed better than 
its $r = 2$ counterpart.}
\label{fig:r2_r3_rho}
\end{center}
\end{figure}

It is also interesting to compare the performance of the most fit $r = 2$ and $r = 3$ rules over multiple generations. Figures~\ref{fig:r2_best_fit} and 
\ref{fig:r3_best_fit} show the best fitness for a sample run for the two neighborhoods as a function of generation. As can be seen in those figures, both trajectories exhibit `epochs' as described by Mitchell \cite{Mitchell:1994:ECA:186092.186116}: There is an initial plateau interrupted by a jagged rise in 
best fitness, where these increases represent the discovery of new, dominant strategies.

\begin{figure} [H]
\begin{center}
\includegraphics[width=\linewidth]{figures/max_epoch_radius2.eps}
\caption{Best fitness for an $r = 2$ run by generation. The maximum fitness trajectory exhibits epochs: the population rapidly leaves epoch 1, flattens out until about generation 30, and then climbs again to another plateau.}
\label{fig:r2_best_fit}
\end{center}
\end{figure}
\begin{figure} 
\begin{center}
\includegraphics[width=\linewidth]{figures/max_epoch_radius3.eps}
\caption{Best fitness for an $r = 3$ run by generation. As in the case of $r = 2$ (fig.~\ref{fig:r2_best_fit}), the trajectory exhibits epochs; in this case, there appear to be three plateaus.}
\label{fig:r3_best_fit}
\end{center}
\end{figure}

Let $\lambda$ be the proportion of \texttt{1}s in the binary representation of a rule \cite{Mitchell:1994:ECA:186092.186116}. Then the transitions 
just described are equivalent to shifting from rules with $\lambda \approx 0$ or $\lambda \approx 1$ to rules with $\lambda \approx \frac{1}{2}$. 
This transition is depicted in 
figures~\ref{fig:histogram_r2} and~\ref{fig:histogram_r3}, for $r = 2$ and $r = 3$ runs, respectively.

\begin{figure} [H]
\begin{center}
\includegraphics[width=\linewidth]{figures/entropy_fitness_plot.eps}
\caption{Average Shannon entropy of transients for 500 elite rules from the 50th generation of the $r = 3$ GA runs, as evaluated over a test set of 
10k ICs. Elite rules do not favor short transients. }
\label{fig:transients}
\end{center}
\end{figure}

\afterpage{%
\begin{figure*}
\begin{center}
\includegraphics[width=\textwidth]{figures/histograms_r2.eps}
\caption{Frequency of different elite rules (as represented by $\lambda$) by generation, for 50 $r = 2$ runs. Generations 1 and 2 are biased towards $\lambda \approx 1.0$ and $\lambda \approx 0.0$ indicating that successful rules converged on all \texttt{0}s or all \texttt{1}s, regardless of $\rho_0$. By generation 15 
successful rules are starting to cluster near $\lambda = 0.5$, and we can also see the characteristic 'dual peaks' resulting from counting both rules that 
began life near $\lambda = 0.0$ and those that began near $\lambda = 1.0$.}
\label{fig:histogram_r2}
\end{center}
\end{figure*}
\clearpage
}

\afterpage{%
\begin{figure*}
\begin{center}
\includegraphics[width=\textwidth]{figures/histograms_r3.eps}
\caption{Frequency of different elite rules (as represented by $\lambda$) by generation, for 50 $r = 3$ runs. Generations 1 and 2 are biased towards $\lambda \approx 1.0$ and $\lambda \approx 0.0$ indicating that successful rules converged on all \texttt{0}s or all \texttt{1}s, regardless of $\rho_0$. By generation 15 
successful rules are starting to cluster near $\lambda = 0.5$, a trend that continues to the final generation.}
\label{fig:histogram_r3}
\end{center}
\end{figure*}
\clearpage
}

Finally, note that each location in the lattice has a transient period in which its state may continue to flip before finally stabilizing (if at all; see figures~\ref{fig:waterfall_1} and \ref{fig:waterfall_2}). One way to quantify this period is to count the number of times a given location changes state 
over the 300 iterations of a rule, given an initial condition. In order to visualize the relation between the length of a transient period and the fitness of a rule, 
we recorded the average Shannon entropy of the transients for each elite rule on a set of 10k test ICs, and plotted this value against the fitness 
for that rule (figure~\ref{fig:transients}).

\subsection{Mutational robustness} \label{sec:2_2}

Mutational robustness occurs when changes in the genotype are phenotypically neutral: The genomic modification results in no significant change 
in the traits expressed, and hence no difference in fitness \cite{wagner_role_2012}. When a phenotype is robust in this way, it may be possible for the 
genome to move to a very different region of search space by accruing changes over multiple generations, increasing the breadth of the search without 
having to cross low-fitness boundaries.

If our CAs are mutationally robust, then the nearby neighbors of elite rules should themselves perform well. To assess whether 
this is the case, we tested the 500 elite $r = 3$ rules from generation 50 against 10k ICs, and selected the best performing rule. 
We then generated 50 sets of 100 mutated rules each from this source rule, with the first set containing mutants with a Hamming distance from the 
source of 1, the second a distance of 2, and so on. The fitness of each mutant was evaluated against 10k ICs; results are given in 
figure~\ref{fig:robustness}.

\begin{figure}
\begin{center}
\includegraphics[width=\linewidth]{figures/mutational_robustness.eps}
\caption{Mutational robustness of the best performing $r = 3$ rule for 0 to 50 mutations. 100 unique variations of the 
original rule are generated with $m$ mutations, and each is evaluated over 10k randomly generated ICs. Mean fitness decreases 
rapidly, but high fitness rules are still present, as indicated by the max fitness values. Within a Hamming distance of $\approx 10$, maximum performance 
actually increases in comparison with the source rule. }
\label{fig:robustness}
\end{center}
\end{figure}

\subsection{In search of a better CA: Biased fitness} \label{sec:2_3}

As noted above, some of the earliest successful strategies to emerge are those that (i) go to all \texttt{1}s or all \texttt{0}s by default, 
yet (ii) change their behavior for values of $\rho$ at the opposite extreme. For example, a rule may for the most part map every 
neighborhood to \texttt{0}, except for neighborhoods with many \texttt{1}s, so an IC with a high enough value of $\rho$ will 
`overwhelm' the default behavior of the rule. In such a rule, every IC with $\rho < 0.5$ is successfully classified, as are ICs with $\rho \approx 1.0$, but 
errors are made on those with $\rho \gtrapprox 0.5$.

Of course, the region around $\rho = 0.5$ is precisely the region we want to encourage the rules to deal with, yet under the fitness testing regime 
used for the experiments reported above, the `easy' cases (where $\rho \approx 1.0$ or $\rho \approx 0.0$) appear in testing sets  
with the same probability as any other IC for the entire GA run. As a result, CAs are still being rewarded for dealing with these easy cases long after their 
utility is exhausted.

To address this issue, we modified the GA so that the fitness gain for successfully classifying ICs is a function of its distance from $\rho = 0.5$: The 
closer to 0.5, the more a correct classification is worth. Furthermore, this function is adjusted depending on the current generation, so that at generation 1 
all ICs are worth the same amount of fitness regardless of their distance from $\rho = 0.5$, and the distance becomes increasingly important as 
the generations progress (see section~\ref{sec:methods}).

We ran 50 runs of the `biased fitness' GA, each for 50 generations, for both $r = 2$ and $r = 3$ cases. Best fitness by generation looks similar to that 
of the unbiased runs reported in section~\ref{sec:2_1}, so we do not give an example here. To find the best performing elite rule after the 
50th generation, we tested each elite rule from each run on a test set of 10k ICs. Figure~\ref{fig:r2_r3_rho_biased} shows the performance 
of the winning biased $r = 2$ and $r = 3$ rules when tested against 1000 ICs for each $0.0 \leq \rho \leq 1.0$ at 0.025 intervals; for purposes of 
comparison, the figure includes the plots from figure~\ref{fig:r2_r3_rho} as well.

\begin{figure}
\begin{center}
\includegraphics[width=\linewidth]{figures/lambda_biased_correctness_plot.eps}
\caption{The best performing elite rule from 50 runs of our `biased fitness' GA, for both $r = 2$ and $r = 3$, plotted against the counterpart rules from 
figure~\ref{fig:r2_r3_rho}.}
\label{fig:r2_r3_rho_biased}
\end{center}
\end{figure}


