\section{Results} \label{results}

For the results reported here, the CA population is 100, selection is tournament with elitism, offspring CA populations are created with both 
mutation and crossover, the IC length is 121, the size of the IC test set is 100, and unless otherwise noted, CAs are run for a maximum of 
200 iterations per IC over 50 generations. When rules are referred to using decimal notation, the encoding follows Mitchell 
\cite{Mitchell:1994:ECA:186092.186116} (rather than Wolfram). See section \ref{sec:methods} 
for full details.

\subsection{Comparing neighborhoods} \label{sec:2_1}
Our first goal was to compare rules with different neighborhoods. In particular, there is a tradeoff between the size of the search space and the 
the amount of information available for purposes of setting a location state. Larger neighborhoods have rules that can 'see' more of the source IC, and 
hence (intuitively) make a more informed decision about how to set the output bit, but this comes at a cost of increasing the size of the search space by multiple 
orders of magnitude ($2^{32}$ vs. $2^{128}$).

Figure~\ref{fig:waterfall} shows two typical CA runs on the same IC. Rule XXXX (left) has such-and-such properties, while rule YYYY is an example of a rule that successfully categorizes the IC within 200 iterations.
\begin{figure}
\begin{center}
\includegraphics[width=\linewidth]{foo.png}
\caption{Waterfall plots of two typical CA runs on the same IC. The top row is the IC, and each successive row is the result of applying the CA rule 
to the preceding row. The left hand figure depicts rule XXXX, which fails to converge within 200 iterations; the right hand figure depicts rule YYYY, 
which correctly categorizes the density of \texttt{1}s in the IC by converging to all \texttt{1}s.}
\label{fig:waterfall}
\end{center}
\end{figure}

Figures~\ref{fig:r2_rho} and \ref{fig:r3_rho} plot the performance of an elite $r = 2$ and an elite $r = 3$ rule, respectively, as evolved after 50 generations. As can be seen, performance did not differ significantly; or maybe it did. We'll find out, and put a summary regarding which rule performs better here.

\begin{figure}
\begin{center}
\includegraphics[width=\linewidth]{foo.png}
\caption{Performance of an elite, $r = 2$ rule as a function of $\rho$. Rule ZZZZ was tested against multiple ICs (when possible), and the precent correct is plotted on the \textit{y}-axis. As expected, the rule performs well when the proportion of \texttt{1}s is 
low or high, but performance drops as $\rho$ approaches the difficult cases around $\frac{1}{2}$.}
\label{fig:r2_rho}
\end{center}
\end{figure}
\begin{figure} [h]
\begin{center}
\includegraphics[width=\linewidth]{foo.png}
\caption{Performance of an elite, $r = 3$ rule as a function of $\rho$. As in the case of an elite $r = 2$ rule (fig.~\ref{fig:r2_rho}), the rule performs well when the proportion of \texttt{1}s is 
low or high, but performance drops as $\rho$ approaches the difficult cases around $\frac{1}{2}$.}
\label{fig:r3_rho}
\end{center}
\end{figure}

It is also interesting to compare the performance of the most fit $r = 2$ and $r = 3$ rules over multiple generations. Figures~\ref{fig:r2_best_fit} and 
\ref{fig:r3_best_fit} show the best fitness for the two neighborhoods as a function of generation. As can be seen in those figures, both trajectories exhibit 
`epochs' as described by Mitchell \cite{Mitchell:1994:ECA:186092.186116}: Plateaus in best fitness are punctuated by moments where a new strategy 
is discovered and becomes dominant in the population. However, as can also be seen from the figures, the $r = 2$ population discovers later epochs sooner 
than the $r = 3$ population - or maybe they don't, or maybe it's the other way around.

\begin{figure}
\begin{center}
\includegraphics[width=\linewidth]{foo.png}
\caption{Best fitness for $r = 2$ CA by generation. The trajectory exhibits epochs in the sense that plateaus in best fitness are punctuated by moments 
where a new type of strategy is discovered, leading to a new plateau.}
\label{fig:r2_best_fit}
\end{center}
\end{figure}
\begin{figure} [h]
\begin{center}
\includegraphics[width=\linewidth]{foo.png}
\caption{Best fitness for $r = 3$ CA by generation. As in the case of $r = 2$ (fig.~\ref{fig:r2_best_fit}), the trajectory exhibits epochs. In contrast to $r = 2$, 
however, it takes longer or shorter or whatever if we see any differences.}
\label{fig:r3_best_fit}
\end{center}
\end{figure}



\subsection{Mutational robustness} \label{sec:2_2}

In this section we consider whether our CAs exhibit mutational robustness as described in \cite{wagner_role_2012}.

\subsection{In search of a better CA} \label{sec:2_3}

In this section we try to extend the results.



