
\section{Methods} \label{sec:methods}

In this section we briefly describe the methods used to generate our results.

\subsection{GA parameters}

All runs of the GA used a population of 50 CAs, and the population is evolved for 50 generations. 
Initial rules were chosen from a uniform distribution of $\lambda \in \lbrack0.0, 1.0\rbrack$.

All CAs begin with an initial fitness of 0.0. At each generation, each CA was tested against a set of 100 ICs by 
running the CA on each IC for 200 iterations or until there was no change in the bit string over two iterations (indicating that 
the CA had settled into a steady state for that IC). 

For the results reported in section~\ref{sec:2_1}, we used the fitness function given in equation~\ref{eq:fitness_1}, where 
$\Delta_{fitness}$ is the change in fitness for a CA, and $\rho_n$ is the iteration at which the CA stopped ($n \leq 200$). This fitness 
function is very simple: Given an IC, the fitness for a CA is increased by 1.0 if and only if it settles to all \texttt{1}s when $\rho_0 > 0.5$ or 
settles to all \texttt{0}s when $\rho_0 < 0.5$.
\begin{equation} \label{eq:fitness_1}
  \Delta_{fitness} =
  \begin{cases}
    1.0 & \text{if } \rho_0 < 0.5 \text{ \& } \rho_n = 0.0,
    \\
    1.0 & \text{if } \rho_0 > 0.5 \text{ \& } \rho_n = 1.0,
   \\
    0.0 & \text{otherwise}.
  \end{cases}
\end{equation}
With fitness function (\ref{eq:fitness_1}), the maximum fitness for a CA during its lifetime is 100, and the minimum is 0. Note that fitness 
scores are reset to 0 after each generation, even for the elite rules.

For the results reported in section~\ref{sec:2_3}, we adjusted the fitness function so that easy-to-classify cases (where 
$\rho_0 \approx 0.0$ or $\rho_0 \approx 1.0$) became worth less in comparison to ICs with $\rho \approx 0.5$ as the generation 
increased (figure~\ref{eq:fitness_2}).
\begin{equation} \label{eq:fitness_2}
  \Delta_{fitness} =
  \begin{cases}
    1.0 + g*(\frac{1}{|\rho - 0.5| * 50}) & \text{if } \rho_0 < 0.5 \text{ \& } \rho_n = 0.0,
    \\
    1.0 + g*(\frac{1}{|\rho - 0.5| * 50})& \text{if } \rho_0 > 0.5 \text{ \& } \rho_n = 1.0,
   \\
    0.0 & \text{otherwise}.
  \end{cases}
\end{equation}
In (\ref{eq:fitness_2}), $g$ is the current generation number, and 50 is a scaling term.

The GA uses tournament selection with elitism. After assessing the fitness of each CA, the top 10\% are copied to the next generation, 
without any mutation or crossover. The remaining 90\% are filled in by randomly selecting two pairs from the population, and the members of 
each pair compete for the opportunity to reproduce, the CA with the higher fitness wins. A single crossover point is chosen at random, 
and a new rule is constructed using one segment from each parent. Finally, 10\% of the bits in the resulting rule are mutated (i.e., flipped).

\subsection{Choosing ICs}

Each generation was tested against a set of 100 ICs, and a new set was created for each generation. ICs were chosen 
from a uniform distribution $\rho_0 \in \lbrack1.0, 0.0\rbrack$ to make the inclusion of easier ICs equally likely as 
harder, $\rho_0 \approx 0.5$ ICs.

We considered changing this distribution for \ref{sec:2_3}: Including the easier-to-classify ICs in the training set 
slows down the discovery of better solutions at later generations, so one way to correct for this is to make the inclusion of 
those ICs less likely as the generation increases. However, we instead opted for adjusting the fitness function as described 
above.

\subsection{Implementation and Optimization}

The GA was implemented in Java with various optimizations. 

All experiments were run on ??????????????

\subsection{Data analysis methods}

I'm not sure what's supposed to go here.

