\section{Discussion} \label{discussion}

There are a number of interesting observations regarding the results reported above; here we survey these 
conclusions.

\subsection{On the difference between radius 2 and 3}

The size of the search space grows with the radius of a rule ($2^{32}$ for $r = 2$, and $2^{128}$ for $r = 3$), suggesting 
that searches may be more difficult for CAs with larger neighborhoods. However, a larger radius also means that the CA has more information 
to work with in deciding which state to transition to, suggesting that finding a solution may be easier for larger radii CAs despite the increased 
search size.

For both $r = 2$ and $r = 3$, we observe the pattern reported in \cite{Mitchell:1994:ECA:186092.186116}: Strategies begin simple and become 
more nuanced over time. We can see this progression in the series of histograms given in figures~\ref{fig:histogram_r2} 
and \ref{fig:histogram_r3}. In both cases, early populations (before generation 5) are dominated with rules that take an IC to all \texttt{0}s or 
all \texttt{1}s. This is a sort of `gambling' strategy because it `bets' that the correct answer 
is one way or the other. Interestingly, we also found rules that gambled in a different way: Oscillate $\rho$ back and forth from 1.0 to 0.0 in the hopes 
that, after 300 iterations, the final global state will be correct. Gambling is better than doing nothing, regardless of how you do it, and the GA 
was good at discovering this.

In both $r = 2$ and $r = 3$, $\lambda$ gradually moves towards 0.5: Rules retain the initial strategy of always converging one way or the other, but begin to include `extreme' values of $\rho_0$ at the other end of the spectrum. In this case we see a slight difference in the histograms for $r = 2$ and 
$r = 3$. In the former, we see the characteristic `dual peak` around $\lambda = 0.5$; this results from the fact that rules converge from both directions, 
i.e., from $\lambda \approx 0.0$ and $\lambda \approx 1.0$, but in the $r = 3$ case, we do not see this pattern; instead, we find a peak around 0.5 
with troughs on either side. We are not certain what is responsible for this distribution, but it could be an artifact of the bin size for the histograms.


\subsection{On mutational robustness}

Here we discuss 2.2.

\subsection{On the quest for a better solution}

Here we discuss 2.3.


